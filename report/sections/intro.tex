Automatic solving of puzzles has become more and more popular because it has applications in various fields ranging from the reassembly of archaeological relics to DNA modelling \cite{robust}. It has been the subject of various studies aiming at the automatic reconstruction via image processing. While most algorithms are designed to solve puzzles composed of square, digitally scrambled pieces, some can even start from pictures of actual puzzles (for example in \cite{shape}). 

In this work, puzzles of coloured square pieces are considered. No prior information from the original image is used, but the pieces are assumed to be correctly oriented. For theses assumptions, solvers with more than 90\% accuracy exist. Different methods of the measure of the dissimilarity between two pieces from the literature are first detailed, then the placement algorithm on the basis of this compatibility is introduced. Finally, the performance of the different methods are compared.